\documentclass{VUMIFPSkursinis}
\usepackage{algorithmicx}
\usepackage{algorithm}
\usepackage{algpseudocode}
\usepackage{amsfonts}
\usepackage{amsmath}
\usepackage{bm}
\usepackage{caption}
\usepackage{color}
\usepackage{float}
\usepackage{graphicx}
\usepackage{listings}
\usepackage{subfig}
\usepackage{wrapfig}
\usepackage{multirow}
\usepackage{longtable}
\usepackage{array,makecell}
\usepackage{enumitem}

% Titulinio aprašas
\university{Vilniaus universitetas}
\faculty{Matematikos ir informatikos fakultetas}
\department{Programų sistemų katedra}
\papertype{Programų sistemų inžinerija}
\title{Keleivių kontrolės vengimo programėlė}
\titleineng{AVOID A TICKET}
\status{2 kurso 5 grupės studentai}
\author{Elena Reivytytė}
\secondauthor{Matas Šilinskas}
\thirdauthor{Kasparas Taminskas}
\fourthauthor{Aidas Vaikšnoras}
\fifthauthor{Tadas Žaliauskas}
\supervisor{dr. Vytautas Valaitis}
\date{Vilnius – \the\year}

% Nustatymai
% \setmainfont{Palemonas}   % Pakeisti teksto šriftą į Palemonas (turi būti įdiegtas sistemoje)
\bibliography{bibliografija}

\begin{document}
\maketitle

\sectionnonum{Anotacija}

\tableofcontents

\sectionnonum{Žodynas}
\begin{enumerate}
	\item Aplikacija - tai programėlė leidžianti naudotis aprašytomis funkcijomis.
	\item Naudotojas - tai bendrinis aplikacijos profilis.
	\item Žaidimas - stalo žaidimo įvykis. (Game)
	\item Vartotojas - galutinis aplikacijos vartotojas, galintis sukurti ar prisijungti prie egzistuojančio žaidimo. (User)
	\item Administratorius - tai žmogus kuris prižiūri korektišką aplikacijos veikimą, turi priėjimą prie statistinės informacijos.
	\item Kompanija - verslo subjektas, galintis aplikacijoje sukurti specialus renginius savo atstovaujamiems žaidimams.
	\item Draugas - tai vartotojas esantis draugų saraše, kuriam galima parašyti žinutę arba pakviesti žaisti konkretų žaidimą. (Friend)
	\item Perspektyva - (Point of view) aplikacijos naudojimo kampas priklausantis nuo naudotojo tipo.
\end{enumerate} 

\section{Įvadas}

\subsection{Programų sistemos pavadinimas}
\subsection{Dalykinė sritis}
\subsection{Probleminė sritis}
Programėlė yra skirta padėti klientams sutrumpinti kelionių laiką parodant, kuriose vietose keleivių kontrolė tikrina autobusus.
\subsection{Vartotojas}
Klientas - bet kokio amžiaus žmogus, kuris turi išmanųjį telefoną ir naudojasi miestinių autobusų teikiamomis paslaugomis.
\subsection{Darbo pagrindas}

\textbf{Darbo pasiskirstymas}\\
\begin{tabular}{lr}
   Elena Reivytytė & 20\% \\
   Aidas Vaikšnoras & 20\% \\
   Kasparas Taminskas & 20\% \\
   Tadas Žaliauskas & 20\% \\
   Matas Šilinskas & 20\% \\
\end{tabular}

\section{Poreikiai}
\begin{enumerate}
	\item Vartotojas gali prisijungti, atsijungti ir registruotis prie programėlės.
	\item Vartotojas turi galimybė prisijungti naudojantis “Facebook”.
	\item Vartotojas gali keisti registracijos metu įvestus duomenis: vardą, pavardę, el. paštą, slaptažodį, prisijungimo vardą.
	\item Administratorius gali keisti kiekvieno vartojo vardą, pavardę, el. paštą, slaptažodį, prisijungimo vardą.
	\item Administratorius gali pridėti naują vartotoją ir administratorių.
	\item Vartotojas gali pridėti žymeklį, nurodantį vietą, kurioje yra kontrolė netoliese nuo jo esančiu atstumu.
	\item Vartotojas gali matyti kitų vartotojų žymeklius, sudėtus žemėlapyje.
	\item Padėti žymekliai rodomi tik tam tikrą laiko tarpą.
	\item Vartotojas gali balsuoti ar žymeklis yra klaidingas ar ne.
	\item Vartotojas gali palikti komentarą prie žymeklio.
	\item Administratorius gali ištrinti visus žymeklius.
	\item Vartotojas gali perskaityti D.U.K. programėlėje.
	\item Vartotojas gali užduoti klausimą.
	\item Programėle turi būti anglų kalba. 
	\item Išjungus programėlę, vartotojas liekas prisijungęs.
\end{enumerate} 

\section{Reikalavimai}
\subsection{Funkciniai reikalavimai }
Šiame skyriuje pateikiami sistemos funkciniai reikalavimai, t.y. pagrindinės sistemos atliekamos funkcijos, konkretūs jų aprašymai.

\newcounter{frcount}
\newcommand\rownumberfr{\stepcounter{frcount}\arabic{frcount}}

\begin{longtable}{ | >{\centering}m{2cm} | m{10cm} | >{\centering}m{2.5cm} | } \caption{Funkciniai reikalavimai.} \endhead \hline
\multicolumn{3}{ |l| }{\textbf{Funkciniai reikalavimai:}} \tabularnewline \hline
\textbf{Numeris} & \centering{\textbf{Reikalavimas}} & \textbf{Svarba} \tabularnewline \hline

FR\rownumberfr & Įsijungęs programėlę, vartotojas gali:
						\begin{enumerate}
							\item registruotis
							\item prisijungti
							\item neteisingai įvedęs duomenis, vartotojas gali pasitaisyti
						\end{enumerate}
				\textit{Žr. skyrius poreikiai, 1 punktas} & Būtina\tabularnewline \hline

FR\rownumberfr & Prisijungęs vartotojas gali atsijungti.\newline \textit{Žr. skyrius poreikiai, 1 punktas} & Būtina\tabularnewline \hline
FR\rownumberfr & Vartotojas gali prisijungti naudodamas “Facebook” paskyrą.\newline \textit{Žr. skyrius poreikiai, 2 punktas} & Būtina\tabularnewline \hline
FR\rownumberfr & Savo paskyroje vartotojas gali keisti:
						\begin{enumerate}
							\item slaptažodį
							\item vardą
							\item pavardę
							\item prisijungimo vardą
							\item el. paštą
						\end{enumerate}
				\textit{Žr. skyrius poreikiai, 3 punktas} & Būtina\tabularnewline \hline
FR\rownumberfr & Administratorius gali pakeisti bet kurio vartotojo:
						\begin{enumerate}
							\item slaptažodį
							\item vardą
							\item pavardę
							\item prisijungimo vardą
							\item el. paštą
						\end{enumerate}
				\textit{Žr. skyrius poreikiai, 4 punktas} & Būtina\tabularnewline \hline
FR\rownumberfr & Administratorius gali pridėti naują vartotoją arba administratorių.\newline \textit{Žr. skyrius poreikiai, 5 punktas} & Būtina\tabularnewline \hline
FR\rownumberfr & Vartotojas gali pridėti žymeklį, nurodantį vietą, kurioje yra kontrolė netoliese nuo jo esančiu atstumu.
				 \newline \textit{Žr. skyrius poreikiai, 6 punktas} & Būtina\tabularnewline \hline
FR\rownumberfr & Vartotojas gali peržiūrėti kitų vartototojų padėtus žymeklius.\newline \textit{Žr. skyrius poreikiai, 7 punktas} & Būtina\tabularnewline \hline
FR\rownumberfr & Padėti žymekliai rodomi 1h 30min skaičiuojant nuo padėjimo laiko, o vėliau yra automatiškai ištrinami.\newline \textit{Žr. skyrius poreikiai, 8 punktas} & Būtina\tabularnewline \hline
FR\rownumberfr & Vartotojas gali balsuoti ar žymeklis yra klaidingas ar ne.\newline \textit{Žr. skyrius poreikiai, 9 punktas} & Būtina\tabularnewline \hline
FR\rownumberfr & Vartotojas gali palikti komentarą prie žymeklio.\newline \textit{Žr. skyrius poreikiai, 10 punktas} & Būtina\tabularnewline \hline
FR\rownumberfr & Administratorius gali ištrinti visus žymeklius.\newline \textit{Žr. skyrius poreikiai, 11 punktas} & Būtina\tabularnewline \hline
FR\rownumberfr & Vartotojas gali perskaityti D.U.K. pačioje programėlėje.\newline \textit{Žr. skyrius poreikiai, 12 punktas} & Būtina\tabularnewline \hline
FR\rownumberfr & Vartotojas gali užduoti klausimą pačioje programėlėje.\newline \textit{Žr. skyrius poreikiai, 13 punktas} & Būtina\tabularnewline \hline
\end{longtable}

\subsection{Nefunkciniai reikalavimai}
Šiame skyriuje pateikiami nefunkciniai reikalavimai sistemoms, t.y. reikalavimai, tiesiogiai nesusiję su sistemos atliekamomis funkcijomis.

\newcounter{nfrcount}
\newcommand\rownumber{\stepcounter{nfrcount}\arabic{nfrcount}}

\subsubsection{OS reikalavimai}
\begin{longtable}{ | >{\centering}m{2cm} | m{10cm} | >{\centering}m{2.5cm} | } \caption{Nefunkciniai OS reikalavimai} \endhead \hline
\multicolumn{3}{ |l| }{\textbf{OS reikalavimai:}} \tabularnewline \hline
\textbf{Numeris} & \centering{\textbf{Reikalavimas}} & \textbf{Svarba} \tabularnewline \hline
NFR\rownumber & Programėlė turi būti palaikoma Android (nuo 4.0 versijos) įrenginiuose & Būtina\tabularnewline \hline
NFR\rownumber & Programėlė palaikoma iOS (nuo 8.0 versijos) įrenginiuose. & Pageidautinas\tabularnewline \hline
NFR\rownumber & Aplikacija turi būti pasiekiama ir be išmanaus telefono - per naršyklę & Būtina\tabularnewline \hline
NFR\rownumber & Išmanusis renginys turi turėti GPS modulį. & Būtina\tabularnewline \hline
NFR\rownumber & Išmanusis įrenginys turi turėti prieigą prie interneto & Būtina\tabularnewline \hline
\end{longtable}

\subsubsection{Saveikos su  kitomis programomis reikalavimai}
\begin{longtable}{ | >{\centering}m{2cm} | m{10cm} | >{\centering}m{2.5cm} | } \caption{Nefunkciniai saveikos su  kitomis programomis reikalavimai} \endhead \hline
\multicolumn{3}{ |l| }{\textbf{Saveikos su  kitomis programomis reikalavimai:}} \tabularnewline \hline
\textbf{Numeris} & \centering{\textbf{Reikalavimas}} & \textbf{Svarba} \tabularnewline \hline
NFR\rownumber & Facebook ir Google api vartotojo autorizacijai & Būtina\tabularnewline \hline
NFR\rownumber & Aplikacija reikalauja GPS prieigos teisių. & Būtina\tabularnewline \hline
NFR\rownumber & Aplikacija reikalauja priegos prie mobiliojo interneto, jei neprisijungta prie Wi-Fi. & Būtina\tabularnewline \hline
NFR\rownumber & Google api žaidimo vietos nustatymui. & Būtina\tabularnewline \hline
\end{longtable}

\subsubsection{Programavimo aplinkos reikalavimai}
\begin{longtable}{ | >{\centering}m{2cm} | m{10cm} | >{\centering}m{2.5cm} | } \caption{Nefunkciniai programavimo aplinkos reikalavimai} \endhead \hline
\multicolumn{3}{ |l| }{\textbf{Programavimo aplinkos reikalavimai:}} \tabularnewline \hline
\textbf{Numeris} & \centering{\textbf{Reikalavimas}} & \textbf{Svarba} \tabularnewline \hline
NFR\rownumber & Programėlė kuriama C\# programavimo kalba. & Būtina\tabularnewline \hline
NFR\rownumber & Kodo versijavimui ir dalinimuisi naudojama Github repozitorija. & Būtina\tabularnewline \hline
\end{longtable}

\subsubsection{Tikslumo reikalavimai}
\begin{longtable}{ | >{\centering}m{2cm} | m{10cm} | >{\centering}m{2.5cm} | } \caption{Nefunkciniai tikslumo reikalavimai duomenų saugojimui} \endhead \hline
\multicolumn{3}{ |l| }{\textbf{Tikslumo reikalavimai duomenų saugojimui:}} \tabularnewline \hline
\textbf{Numeris} & \centering{\textbf{Reikalavimas}} & \textbf{Svarba} \tabularnewline \hline
NFR\rownumber & Vartotojo informacijai skiriama:
						\begin{enumerate}
							\item vardui, pavardei ir prisijungimo vardui - 15 simbolių
							\item slaptažodžiui - 20 simbolių
							\item el. paštui - 60 simbolių
						\end{enumerate}
			  & Būtina\tabularnewline \hline
NFR\rownumber & Vartotojo prisijungimo vardas ir slaptažodis turi susidaryti bent iš 5 simbolių. & Būtina\tabularnewline \hline
NFR\rownumber & Kiekvieno vartotojo slaptažodis privalo turėti bent vieną didžiąją raidę ir bent vieną skaičių. & Būtina\tabularnewline \hline
NFR\rownumber & Vartotojo prisijungimo vardas turi būti unikalus. & Būtina\tabularnewline \hline
NFR\rownumber & Programėlėje esantis tekstas rašomas anglų kalba. \newline \textit{Žr. skyrius poreikiai, 14 punktas} & Būtina\tabularnewline \hline
\end{longtable}

\subsubsection{Aptarnavimo ir priežiūros reikalavimai}
\begin{longtable}{ | >{\centering}m{2cm} | m{10cm} | >{\centering}m{2.5cm} | } \caption{Nefunkciniai aptarnavimo ir priežiūros reikalavimai} \endhead \hline
\multicolumn{3}{ |l| }{\textbf{Aptarnavimo ir priežiūros reikalavimai:}} \tabularnewline \hline
\textbf{Numeris} & \centering{\textbf{Reikalavimas}} & \textbf{Svarba} \tabularnewline \hline
NFR\rownumber & Į vartotojo užduodamus klausimus darbo metu turi būti atsakyta ne vėliau nei per valandą, o ne darbo metu ne vėliau nei per 12 valandų. & Pageidautina\tabularnewline \hline
NFR\rownumber & Praplėtus sistemos funkcionalumą, būtina testuoti atnaujinimus, kad būtų užtikrinta programėlės sklandi veikla, prieš leidžiant jais naudotis registruotiems vartotojams, svečiams ir administratoriui. & Būtina\tabularnewline \hline
NFR\rownumber & Testai turi padengti 80\% kodo. & Būtina\tabularnewline \hline
NFR\rownumber & Programėlę galima atsinaujinti per Google Play. & Būtina\tabularnewline \hline
\end{longtable}

\subsubsection{Dalykinės srities metaforų reikalavimai}
\begin{longtable}{ | >{\centering}m{2cm} | m{10cm} | >{\centering}m{2.5cm} | } \caption{Nefunkciniai dalykinės srities metaforų reikalavimai} \endhead \hline
\multicolumn{3}{ |l| }{\textbf{Dalykinės srities metaforų reikalavimai:}} \tabularnewline \hline
\textbf{Numeris} & \centering{\textbf{Reikalavimas}} & \textbf{Svarba} \tabularnewline \hline
NFR\rownumber & Žymeklių duomenys (koordinates ir padėjimo laikas) saugomi duomenų bazėje. & Būtina\tabularnewline \hline
NFR\rownumber & Vartotojas gali padėti ne daugiau 10 žymeklių per dieną. & Būtina\tabularnewline \hline
\end{longtable}
			
\subsection{Rezultatai}
Šioje dalyje pristatyta mobiliosios stalo žaidimų programėlės Board Games 4+1 architektūra ir
modelis, kurį sudaro loginis, kūrimo, fizinis, procesų ir užduočių pjūviai, atskleidžiantys detalų
ir konkretų projektuojamos sistemos vaizdą.

\sectionnonum{Išvada}
Visos trys projekto dalys, atliktos Programų sistemų inžinerijos kurso metu, leido geriau susipažinti
su vientisu ir nuosekliu IT projekto kūrimo procesu, pradedant nuo detalios verslo srities analizės,
į kurią įeina ir rinkos tyrimai, tęsiant reikalavimų specifikacija, kuri nusako aiškius numatomos
sistemos rėmus ir baigiant aiškiu ir detaliu architektūriniu vaizdu, kurį apibrėžia skirtingi sistemos
architektūriniai pjūviai. Nuoseklus projekto kūrimo procesas, kurio pagal šio kurso turinį buvo bandoma
laikytis, leidžia išvengti netikėtumų eigoje, papildomų nenumatytų kaštų, tobulinimo kliūčių ir labai
tikėtina - greitos sistemos ,,žūties''.
	
\sectionnonum{Literatūros sąrašas}
\begin{itemize}
	\item Doc. dr. K. Petrausko Programų Sistemų Inžinerijos kurso konspektai
    \item A. Abran, J. W. Moore, P.Bourque, R. Dupuis, L. L. Tripp - ,,Guide to the Software Engineering Body of Knowledge''
	\item UML dokumentacija \url{https://www.tutorialspoint.com/uml/uml_2_overview.htm}
	\item OMG UML v.2.5 Dokumentacija diagramoms, žymėjimui
	\item Informacija apie išdėstymo tinkle (Deployment) diagramas, pavyzdžiai \url{http://www.uml-diagrams.org/deployment-diagrams-overview.html} 
	\item https://osp.stat.gov.lt/statistiniu-rodikliu-analize?theme=all\#/
	\item https://www.androidauthority.com/publishing-first-app-play-store-need-know-383572/
	\item https://www.serveriai.lt/
	\item https://www.ovh.com/us/dedicated-servers/
	\item https://adwords.google.com/ko/KeywordPlanner/
	\item https://www.webpagefx.com/SEO-Pricing.html
\end{itemize}

		
\end{document}
